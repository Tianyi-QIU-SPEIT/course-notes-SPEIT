\documentclass{article}

\usepackage{graphicx} 
\usepackage[french]{babel}
\usepackage[T1]{fontenc}
\usepackage[utf8]{inputenc}
\usepackage{lmodern}
\usepackage{microtype}
\usepackage{hyperref}
\usepackage{amsmath}
\usepackage{amssymb}
\usepackage{geometry}
\usepackage{fancyhdr}
\usepackage{ctex}
\usepackage{tcolorbox}
\pagenumbering{arabic}
\pagestyle{fancy}
\fancyhead[L]{École d'Ingénieurs Paris-SJTU}
\fancyhead[R]{Corentin邱天意}
\fancyfoot[C]{\thepage}
\renewcommand{\headrulewidth}{1pt}
\renewcommand{\footrulewidth}{1pt}

\makeatletter
\@addtoreset{section}{part}
\def\@part[#1]#2{%
    \ifnum \c@secnumdepth >\m@ne
      \refstepcounter{part}%
      \addcontentsline{toc}{part}{\thepart\hspace{1em}#1}%
    \else
      \addcontentsline{toc}{part}{#1}%
    \fi
    {\parindent \z@ \raggedright
     \interlinepenalty \@M
     \normalfont\raggedright
     \ifnum \c@secnumdepth >\m@ne
       \LARGE\bfseries \partname\nobreakspace\thepart
       \par\nobreak
     \fi
     \huge \bfseries #2%
     \markboth{}{}\par}%
    \nobreak
    \vskip 3ex
    \@afterheading}
\renewcommand\partname{Topic}
\makeatother

\title{\textbf{Notes du Cours : MATH2308P}\\ Cours assuré par Sébastien GODILLON}
\author{Fiche \LaTeX \space rédigé par Corentin邱天意}
\date{Semestre 2024-2025-2}

\begin{document}

\maketitle

\centerline{\includegraphics[scale=0.4]{sjtu}}

\newpage
\tableofcontents

\newpage

\part{Espaces vectoriels normés}

On commence notre travail avec les espaces vectoriels normés. 

Dans tout ce chapitre, $E$ désigne un $\mathbb{K}$-espace vectoriel, où $\mathbb{K} = \mathbb{R}$ ou $\mathbb{C}$, et on note $0_E$ le vecteur nul de $E$.



\section{Cours 18 février: Normes et distances}

\begin{tcolorbox}[colback=red!5!white,colframe=red!75!black,title=Définition 1.1]

Une \textbf{norme} sur $E$ est une application de $E$ dans $\mathbb{R}$. Elle a pour notation $N$, et elle vérifie les propriétés suivantes:

\begin{itemize}
 \item $\forall x \in E, N(x)=0 \Rightarrow x = 0_E$  \hfill (séparation)
 \item $\forall (\lambda, x) \in \mathbb{K} \times E, N(\lambda x) = |\lambda|N(x)$  \hfill (homogénéité absolue)
 \item $\forall (x,y) \in E^2, \quad N(x+y) \leq N(x) + N(y)$ \hfill (sous-additivité)
 
\end{itemize}

\tcblower

Dans le deuxième point, $|\lambda|$ peut représenter la valeur absolue(en $\mathbb{R}$) ou le module(en $\mathbb{C}$), et \c ca dépend de l'ensemble dans lequel on se place.

\end{tcolorbox}

\begin{tcolorbox}[colback=red!5!white,colframe=red!75!black,title=Définition 1.2]

Si $N$ est une norme sur $E$, alors on dit que $(E,N)$ est un \textbf{espace vectoriel normé}.

\end{tcolorbox}

\begin{tcolorbox}[colback=blue!5!white,colframe=blue!75!black,title=Proposition 1.1]

Soit $N$ une norme sur $E$, alors on a:

\begin{itemize}
    \item $N(0_E) = 0$ \hfill (réciproque de la séparation)
    \item $\forall x \in E, \quad N(x) \geq 0$ \hfill (positivité)
    \item $\forall (x,y) \in E^2, \quad |N(x) - N(y)| \leq N(x - y)$ \hfill (``continuité'')
    
\end{itemize}

\tcblower

Petit remarque 1: la première nous donne l'équivalence dans la propriété de séparation: $\forall x \in E, N(x) = 0 \Longleftrightarrow x = 0_E$

Petit remarque 2 : dans la troisième, $|N(x) - N(y)|$ désigne la valeur absolue puisque la norme est une application dans $\mathbb{R}$.

\end{tcolorbox}

\paragraph{Preuve}: Soient $(x, y) \in E^2$.
\begin{itemize}
 \item $N(0_{E}) = N(0.x) = |0| N(x) = 0$, donc on a: \underline{$N(0_E) = 0$}.
          \newline Remarque: ne mélangez pas $0_E$ et $0$.
 \item D'après la propriété qu'on vient de démontrer, on a:
          \[
          0 = N(0_{E}) = N(x-x) = N(x+(-x)) 
          \]
          De plus, par sous-additivité, on a: 
          \[
          N(x+(-x)) \leq N(x) + N(-x) = N(x) + |-1|N(x) = 2N(x)
          \]
          On obtient \underline{$N(x) \geq 0$} en mettent les deux relations ensemble.
 \item Rappel: $|x| \geq k \Longleftrightarrow -k \leq x \leq k$. 
          \newline Donc il faut démontrer les inégalités à gauche et à droite.
          \begin{itemize}
          \item $N(x) = N(x - y + y) \leq N(x - y) + N(y)$ (par sous-additivité), et on trouve la relation $N(x) - N(y) \leq N(x-y)$.
          \item De même fa\c con on trouve l'autre, en utilisant $N(y)$ au début: $-N(x-y) \leq N(x) - N(y)$.
          \end{itemize}
          Ces deux inégalités nous donnent le résultat: $\forall (x,y) \in E^2, \quad |N(x) - N(y)| \leq N(x - y)$.
 
 
\end{itemize}

\begin{tcolorbox}[colback=yellow!5!white,colframe=yellow!75!black,title=Remarque 1.1]

Dans la troisième on reconnaît une propriété de continuité. Si $x$ est proche de $y$ (``tend vers''), alors $x-y$ est proche du vecteur nul. Donc $N(x,y)$ devient proche de 0, $|N(x)-N(y)|$ aussi(par séparation). Donc $N(x)$  est proche de $N(y)$.



\end{tcolorbox}


\begin{tcolorbox}[colback=cyan!5!white,colframe=cyan!75!black,title=Exemple 1.1]

\textbf{La valeur absolue de $\mathbb{R}$}

On dit que l'application $N: x \mapsto |x|$ est une norme sur $\mathbb{R}$, parce qu'elle vérifie les conditions:

\begin{itemize}
 \item $\forall x \in \mathbb{R}, |x| \leq 0 \Longleftrightarrow x = 0$.
 \item $\forall (\lambda, x) \in {\mathbb{R}}^{2}, |\lambda x| = |\lambda||x|$.
 \item $\forall (x, y) \in {\mathbb{R}}^{2}, |x+y| \leq |x|+|y|$(l'inégalité triangulaire).
\end{itemize}

\textbf{Le module de $\mathbb{C}$}

De même, $N: x \mapsto |x|$ est une norme sur $\mathbb{C}$.

On peut remarquer que $(\mathbb{K}, |.|)$ est un espace vectoriel normé.

\end{tcolorbox}

\begin{tcolorbox}[colback=yellow!5!white,colframe=yellow!75!black,title=Remarque 1.2]

Les normes sont les objets qui généralisent la valeur absolue et le module pour les espaces vectoriels plus grands que $\mathbb{K}$.

\end{tcolorbox}



\begin{tcolorbox}[colback=gray!5!white,colframe=gray!75!black,title=Rappel 1.1]

Pour qu'on puisse commencer à étudier les distances, on rappelle que:

\begin{itemize}
 \item La valeur absolue du réel $a$ représent la distance entre 0 et $a$ sur la droite réelle.
 \item Même chose pour le module pour les complexes, mais cette fois on trouve la distance sur le plan complexe.
 \item Plus généralement $|a-b|$ représent la distance entre $a$ et $b$.
\end{itemize}

\end{tcolorbox}

\begin{tcolorbox}[colback=red!5!white,colframe=red!75!black,title=Définition 1.3]

Soit $(E,N)$ un espace vectoriel normé, alors l'application:
\[
d : 
\begin{cases} 
E^{2} \to [ 0, +\infty [ \\
(a,b) \mapsto d(a,b) = N(a-b)
\end{cases}
\]
est une \textbf{distance} sur E. De plus, elle vérifie 3 propriétés:


\begin{itemize}
 \item $\forall (a,b)\in E^{2}, d(a,b) = d(b,a)$ \hfill (symétrie)
 \item $\forall (a,b)\in E^{2}, d(a,b) = 0 \Longleftrightarrow a=b$ \hfill (séparation)
 \item $\forall (a,b,c) \in E^{3}, d(a,b) \leq d(a,c) + d(c,b)$ \hfill (inégalité triangulaire)
\end{itemize}

\tcblower

Remarque: la deuxième propriété se démontre avec la séparation des normes, et la troisième avec la homogénéité absolue.



\end{tcolorbox}

\begin{tcolorbox}[colback=blue!5!white,colframe=blue!75!black,title=Propriété 1.1]

Notre espace vectoriel normé $(E,N,d)$ est un \textbf{espace métrique}.

\tcblower
Vous pouvez regarder les autres livres pour la définition d'un espace métrique, qui est essentiellement un ensemble muni d'une distance:).
\end{tcolorbox}






\begin{tcolorbox}[colback=blue!5!white,colframe=blue!75!black,title=Propriété 1.2]

La translation et l'homothétie, on les a vu au MATH1301P, dans le chapitre de la géométrie euclidienne. La norme vérifie aussi ces deux propriétés:

\begin{itemize}
 \item $d$ est \textbf{invariante par translation}, c'est-à-dire: $\forall(t,x,y)\in E^{3}, d(x+t, y+t) = d(x,y)$.
 \item $d$ est \textbf{absoluement homogène par homothétie}, c'est-à-dire: $\forall(\lambda, x, y) \in \mathbb{K} \times E^{2}, d(\lambda x, \lambda y) = |\lambda|d(x,y)$.
\end{itemize}

\end{tcolorbox}

\begin{tcolorbox}[colback=yellow!5!white,colframe=yellow!75!black,title=Remarque 1.3]

On peut aussi faire de la topologie dans les espaces métriques, la théorie est plus générale(car on a pas besoin de la structure d'espace vectoriel).

\tcblower
Les espaces métriques fournissent un cadre plus général que les espaces vectoriels normés pour introduire les différentes notions de topologie de ce chapitre. Cependant, on se limitera ici à I'étude moins abstraite des espaces vectoriels normés afin de pouvoir continuer à utiliser les opérations classigues d'algèbre linéaire.


\end{tcolorbox}





\section{Cours 25 février: }























\end{document}