\documentclass{article}

\usepackage{graphicx} 
\usepackage[french]{babel}
\usepackage[T1]{fontenc}
\usepackage[utf8]{inputenc}
\usepackage{lmodern}
\usepackage{microtype}
\usepackage{hyperref}
\usepackage{amsmath}
\usepackage{amssymb}
\usepackage{geometry}
\usepackage{fancyhdr}
\usepackage{ctex}
\usepackage{tcolorbox}
\pagenumbering{arabic}
\pagestyle{fancy}
\fancyhead[L]{École d'Ingénieurs Paris-SJTU}
\fancyhead[R]{Corentin邱天意}
\fancyfoot[C]{\thepage}
\renewcommand{\headrulewidth}{1pt}
\renewcommand{\footrulewidth}{1pt}

\makeatletter
\@addtoreset{section}{part}
\def\@part[#1]#2{%
    \ifnum \c@secnumdepth >\m@ne
      \refstepcounter{part}%
      \addcontentsline{toc}{part}{\thepart\hspace{1em}#1}%
    \else
      \addcontentsline{toc}{part}{#1}%
    \fi
    {\parindent \z@ \raggedright
     \interlinepenalty \@M
     \normalfont\raggedright
     \ifnum \c@secnumdepth >\m@ne
       \LARGE\bfseries \partname\nobreakspace\thepart
       \par\nobreak
     \fi
     \huge \bfseries #2%
     \markboth{}{}\par}%
    \nobreak
    \vskip 3ex
    \@afterheading}
\renewcommand\partname{Topic}
\makeatother

\title{\textbf{Notes du Cours : MATH2308P}\\ Cours assuré par Sébastien GODILLON}
\author{Fiche \LaTeX \space rédigé par Corentin邱天意}
\date{Semestre 2024-2025-2}

\begin{document}

\maketitle

\centerline{\includegraphics[scale=0.4]{sjtu}}

\newpage
\tableofcontents

\newpage

\part{Espaces vectoriels normés}

On commence notre travail avec les espaces vectoriels normés. 

Dans tout ce chapitre, $E$ désigne un $\mathbb{K}$-espace vectoriel, où $\mathbb{K} = \mathbb{R}$ ou $\mathbb{C}$, et on note $0_E$ le vecteur nul de $E$.


\section{Cours 18 février: Normes et distances}

\begin{tcolorbox}[colback=red!5!white,colframe=red!75!black,title=Définition 1.1]

Une \textbf{norme} sur $E$ est une application de $E$ dans $\mathbb{R}$. Elle a pour notation $N$, et elle vérifie les propriétés suivantes:

\begin{itemize}
 \item $\forall x \in E, N(x)=0 \Rightarrow x = 0_E$  \hfill (séparation)
 \item $\forall (\lambda, x) \in \mathbb{K} \times E, N(\lambda x) = |\lambda|N(x)$  \hfill (homogénéité absolue)
 \item $\forall (x,y) \in E^2, \quad N(x+y) \leq N(x) + N(y)$ \hfill (sous-additivité)
 
\end{itemize}

\tcblower

Dans le deuxième point, $|\lambda|$ peut représenter la valeur absolue(en $\mathbb{R}$) ou le module(en $\mathbb{C}$), et \c ca dépend de l'ensemble dans lequel on se place.

\end{tcolorbox}

\begin{tcolorbox}[colback=red!5!white,colframe=red!75!black,title=Définition 1.2]

Si $N$ est une norme sur $E$, alors on dit que $(E,N)$ est un \textbf{espace vectoriel normé}.

\end{tcolorbox}

\begin{tcolorbox}[colback=blue!5!white,colframe=blue!75!black,title=Proposition 1.1]

Soit $N$ une norme sur $E$, alors on a:

\begin{itemize}
    \item $N(0_E) = 0$ \hfill (réciproque de la séparation)
    \item $\forall x \in E, \quad N(x) \geq 0$ \hfill (positivité)
    \item $\forall (x,y) \in E^2, \quad |N(x) - N(y)| \leq N(x - y)$ \hfill (``continuité'')
    
\end{itemize}

\tcblower

Petit remarque 1: la première nous donne l'équivalence dans la propriété de séparation: $\forall x \in E, N(x) = 0 \Longleftrightarrow x = 0_E$

Petit remarque 2 : dans la troisième, $|N(x) - N(y)|$ désigne la valeur absolue puisque la norme est une application dans $\mathbb{R}$.

\end{tcolorbox}

\paragraph{Preuve}: Soient $(x, y) \in E^2$.
\begin{itemize}
 \item $N(0_{E}) = N(0.x) = |0| N(x) = 0$, donc on a: \underline{$N(0_E) = 0$}.
          \newline Remarque: ne mélangez pas $0_E$ et $0$.
 \item D'après la propriété qu'on vient de démontrer, on a:
          \[
          0 = N(0_{E}) = N(x-x) = N(x+(-x)) 
          \]
          De plus, par sous-additivité, on a: 
          \[
          N(x+(-x)) \leq N(x) + N(-x) = N(x) + |-1|N(x) = 2N(x)
          \]
          On obtient \underline{$N(x) \geq 0$} en mettent les deux relations ensemble.
 \item Rappel: $|x| \geq k \Longleftrightarrow -k \leq x \leq k$. 
          \newline Donc il faut démontrer les inégalités à gauche et à droite.
 
 
\end{itemize}


\end{document}